\newcommand{\person}[1]{{\scshape #1}}
\chapter{Acknowledgments \statusgreen}

\section*{Editing}

Many helped me write this book, but a few that took very active roles in making it happen are due credit. \person{Daniel Nickless} is the man behind all the diagrams, we enjoyed many hours of my giving birth to impromptu visualisations and was happy to redo them several times. Dan is a virtuoso of Illustrator and has also become a latex expert to typeset some nice trees. 

I am hugely indebted to \person{Črt Ahlin} who, beside managing the book project, has also contributed some top quality text.
He also undertook many of the unrewarding tedious jobs of compiling the indexes and glossaries. Both Črt and Dan (a native speaker of English) as well as \person{Attila Lendvai} did an amazing job at proof-reading and correcting mistakes and typos.

Special thanks is due to \person{Edina Lovas} whose support and enthusiasm has always helped push me along. 

\section*{Authors}

Swarm is co-fathered by my revered friend and colleague, the awesome \person{Daniel A. Nagy}.  Daniel invented the fundamental design of Swarm and should get the credit for quite a few major architectural decisions, innovative ideas and formal insights presented in this book. 

\person{Aron Fisher}'s contribution to Swarm would be hard to overstate. Most of what is Swarm now started or got worked out in sessions with Aron.  He not only contributed ingredients overall, but also coauthored the first few orange papers and not least was always at the forefront, presenting and explaining our tech in conferences and meetups.

I thank Daniel and Aron for bearing with me, suffering my sloppy, half-baked ideas, bringing clarity and always understanding the maths.
Without claiming full endorsement or any responsibility for what is written here,
I would consider Daniel and Aron as co-authors of this book.

I owe deep gratitude to my partner in crime \person{Gregor Žavcer} who is basically running the project currently. Gregor's honest fascination and unrelenting dedication to the project is what kept me going through many a low moment in the past. Our shared vision of decentralised future of data economy ranging from technological innovation to ethical direction served as the foundation for our collaboration. A great part of this book got shaped during our night-long brainstorming sessions. Gregor even contributed content in the book.

I would like to thank \person{Rinke Hendriksen}, who contributed significant insight and innovation, mainly in the area of incentivisation. His economic theory background continues to prove invaluable in understanding our incentive design. Deep discussions with him led to new solutions, many improvements and insight. He currently manages development as product owner.

People who not only have a major part in coming up with the ideas but inadvertently contributed actual text in the form of excepts from earlier work are \person{Daniel A. Nagy}, \person{Aron Fisher}. Also \person{Fabio Barone} on incentives,  
\person{Javier Peletier} on feeds and \person{Louis Holbrook} on feeds and pss. 

\section*{Feedback}

The book benefited immensely from feedback. Those who went through the pain of reading early versions and commented on the work in progress giving their criticism or pointing out unclarities deserve to be mentioned: \person{Henning Dietrich}, \person{Brendan Graetz}, \person{Marcelo Ortelli}, \person{Santiago Regusci}, \person{Vojtěch Šimetka} of IOV labs, 
\person{Attila Lendvai}, \person{Abel Bodo}, 
\person{Elad Nachmias}, \person{Janoš Guljaš},  \person{Petar Radović}, \person{Louis Holbrook}, \person{Zahoor Mohamed}.



\section*{Conception and influences}

The book of Swarm is itself an expression of the grand idea of Swarm: [the pursuit of] the vision of decentralised storage and messaging on top of the blockchain. The concept and first formulation of Swarm as one of the pillars of a holy trinity to realise the Decentralised Web appeared before the launch of Ethereum in early 2015. It was by the Ethereum founders \person{Vitalik Buterin}, \person{Gavin Wood} and \person{Jeffrey Wilcke} that Swarm was trolled onto the slippery slope of bee jokes and geek humor. The protocol tags \emph{bzz} and \emph{shh} were both coined by Vitalik. 

People who were close to the cradle of Swarm are \person{Alex Leverington}, \person{Felix Lange} early discussions with whom catalysed fundamental decisions that led to the design of Swarm as it now presents.

The foundations of Swarm were laid over the course of 2015.
Daniel worked with \person{Zsolt Felföldi}, of light client fame, whose code is still being seen here and there in the Go Ethereum-based Swarm implementation. His ideas clearly have a hallmark on what Swarm set out to be. 

We are hugely indebted to \person{Elad Verbin}, who for years has been acting as a scientific as well as strategic advisor to Swarm. Elad put considerable effort into the project, his insight and depth is in all areas of Swarm are unparalleled, his insight regarding the isomorphism between pointer-based functional data structures and content addressed distributed data had a major impact on how we handle higher level functionality. Our work on a swarm interpreter inspired the Swarm script.

Special thanks due to \person{Daniel Varga}, \person{Attila Lendvai}, \person{Attila Gazsó} for long nights of brainstorming, I learnt an awful lot from you. Thanks to \person{Alexey Akhunov}, \person{Jordi Baylina} for major technical insight and inspiration.
My very special friend \person{Anand Jaisingh} deserves my gratitude for his unshakeable trust in me and the project and unique inspiration and synergy that was catalised by his presence in the halo of Swarm.

Early in the project, we spent quite some time with \person{Alex van der Sande}, \person{Fabian Vogelsteller} discussing Swarm and its potentials. Many ideas that came to life as a result, including manifests and the HTTP API owe them credit. People in or around the Ethereum Foundation who shaped the idea of Swarm include \person{Taylor Gerring}, \person{Christian Reitwiessner}, \person{Stephan Tual} and \person{Alex Beregszaszi}, \person{Piper Merriam} and \person{Nick Savers}. 

\section*{Team}

First and foremost, thanks to \person{Jeffrey Wilcke}, one of the three founders, and team lead of the Amsterdam go-ethereum team. He was supporting the Ethersphere subteam Dani, Fefe and me, protecting the project in times of austerity.
I am forever grateful to all current and past members of the Swarm team: ethernal gratitude to \person{Nick Johnson} who, during the brief period he was on the swarm team, created ENS. Thanks to those special ones longest on the team: \person{Louis Holbrook}, \person{Zahoor Mohamed} and \person{Fabio Barone} their creativity and faith helped us through rough  times. Thanks to \person{Anton Evangelatov}, \person{Bálint Gábor}, \person{Janoš Guljaš} and \person{Elad Nachmias} for their massive contribution to the codebase. Thanks to \person{Ferenc Szabó}, \person{Vlad Gluhovsky}, \person{Rafael Matias}f and many that cannot be named but contributed to the code.
\person{Ralph Pichler} deserves a special mention, he has been a keen follower and supporter of our project for many years and gradually became an honorary member, he implemented the initial version of the entire smart contract suite for swap, swear and swindle, and been driving the development of blockchain interaction and key management in the recent year.

Major kudos to \person{Janoš Guljaš} who bravely took over the role of engineering lead and created a new killer team in Belgrade with the excellent \person{Petar Radović}, \person{Svetomir Smiljković} and \person{Ivan Vandot}. 

I am grateful to \person{Vojtěch Šimetka} and the Swarm contingent at IOV labs who basically saved the project, \person{Marcelo Ortelli, Santiago Regusci} are major contributors to the current codebase alongside the  Belgrade team. 

I would also like to thank \person{Tim Bansemer}, who is one of a kind, with unimaginable effectiveness and drive, his contributions will always be felt in and around the team, the code, the documentation, and the processes.

\section*{Ecosystem}

Swarm has always attracted a wide community of enthusiasts as well as an ecosystem of companies and projects whose support and encouragement kept us alive during some dark days. 
\person{Jarrad Hope}, \person{Jacek Sieka}, \person{Oscar Thoren} of Status, \person{Marcin Rabenda} of Consensys, \person{Tadej Fius}, \person{Zenel Batagelj} from Datafund, \person{Javier and Alfonso Peletier} of Epic Labs, \person{Eric Tang} and \person{Doug Petkanics} of LivePeer, \person{Sourabh Niyogi} of Wolk, \person{Vaughn McKenzie}, \person{Fred Tibbles} and \person{Oren Sokolovsky} of JAAK, \person{Carl Youngblood}, \person{Paul Le Cam}, \person{Shane Howley}, \person{Doug Leonard} from Mainframe, \person{Anand Jaisingh}, \person{Dimitry Kholkhov}, \person{Igor Sharudin} of BeeFree, \person{Attila Gazsó} from the  Felfele Foundation. 

I would like to thank all participants of Swarm Summits, hack weeks and other gatherings. Many of the ideas in this book developed as a result of conversations on these events. I want to thank \person{Michelle Thuy}, \person{Kevin Mohanan}.

Finally, I want to express gratitude to my mentors and teachers, \person{András Kornai}, \person{László Kálmán}, \person{Péter Halácsy} and \person{Péter Rebrus} who shaped my thinking and skills and will always be my intellectual heroes.

